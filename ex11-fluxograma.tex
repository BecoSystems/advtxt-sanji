%%%%%%%%%%%%%%%%%%%%%%%%%%%%%%%%%%%%%%%%%%%%%%%%%%%%%%%%%%%%%%%%%%%%%%%%%%%%%%%%%%%%%%%%
% Criação de Fluxograma usando LaTeX
%
% Assunto: Fluxograma de um código de aventura interativa do Sanji - One Piece
%
% Autores:
%     Arthur Rodrigues Rocha Lopes
%     Dante De Sá Leitão Melo De Oliveira
%     Lucas Pereira Silveira De Azevedo
%
% Coordenação:
%     Prof. Dr. Ruben Carlo Benante
%
% Data: 27 de abril de 2025
%%%%%%%%%%%%%%%%%%%%%%%%%%%%%%%%%%%%%%%%%%%%%%%%%%%%%%%%%%%%%%%%%%%%%%%%%%%%%%%%%%%%%%%%


%%%%%%%%%%%%%%%%%%%%%%%%%%%%%%%%%%%%%%%%%%%%%%%%%%%%%%%%%%%%%%%%%%%%%%%%%%%%%%%%%%%%%%%%
% Para gerar o PDF use o comando make com o makefile configurado:
%
%    $ make aventura-sanji.pdf
%
% O conteúdo do makefile é composto dos 3 seguintes comandos que ficam assim automatizados:
%    $ pdflatex aventura-sanji.tex -o aventura-sanji.pdf
%    $ bibtex biblio
%    $ pdflatex aventura-sanji.tex -o aventura-sanji.pdf


%%%%%%%%%%%%%%%%%%%%%%%%%%%%%%%%%%%%%%%%%%%%%%%%%%%%%%%%%%%%%%%%%%%%%%%%%%%%%%%%%%%%%%%%
% preambulo %%%%%%%%%%%%%%%%%%%%%%%%%%%%%%%%%%%%%%%%%%%%%%%%%%%%%%%%%%%%%%%%%%%%%%%%%%%%
\documentclass[a4paper,12pt]{article} %twocolumn
\usepackage[left=2.5cm,right=2cm,top=2.5cm,bottom=2cm]{geometry}
\usepackage[utf8]{inputenc} % letras acentuadas
\usepackage[portuguese]{babel} % tradução de títulos
\usepackage[colorlinks]{hyperref}
\usepackage{tikz} % para adicionar fluxogramas
\usepackage{algorithm} % ambiente para índice de algoritmos
\usepackage{algpseudocode} % fonte e estilo do algoritmo
\usepackage{graphicx} % permite adicionar imagens
\usepackage{indentfirst} % indenta o primeiro parágrafo também
\usepackage{url} % permite adicionar links de URLs e emails

\DeclareUrlCommand\email{\urlstyle{mm}} % comando para email bonito
\floatname{algorithm}{Algoritmo} % tradução da palavra algoritimo no ambiente de índice

\usetikzlibrary{shapes.geometric, shapes.symbols,arrows} % ajuste do tikz para incluir formas e setas

%%%%%%%%%%%%%%%%%%%%%%%%%%%%%%%%%%%%%%%%%%%%%%%%%%%%%%%%%%%%%%%%%%%%%%%%%%%%%%%%%%%%%%%%
% capa %%%%%%%%%%%%%%%%%%%%%%%%%%%%%%%%%%%%%%%%%%%%%%%%%%%%%%%%%%%%%%%%%%%%%%%%%%%%%%%%%
\title{Fluxograma de Aventura Interativa do Sanji - One Piece}
\author{
    Arthur Rodrigues Rocha Lopes \\
    Dante De Sá Leitão Melo De Oliveira \\
    Lucas Pereira Silveira De Azevedo
}
\date{27 de abril de 2025}

\begin{document}

\maketitle

%%%%%%%%%%%%%%%%%%%%%%%%%%%%%%%%%%%%%%%%%%%%%%%%%%%%%%%%%%%%%%%%%%%%%%%%%%%%%%%%%%%%%%%%
% resumo %%%%%%%%%%%%%%%%%%%%%%%%%%%%%%%%%%%%%%%%%%%%%%%%%%%%%%%%%%%%%%%%%%%%%%%%%%%%%%%

\begin{abstract}
\textbf{Assunto:} Fluxograma do código da aventura interativa de Sanji.

Este projeto descreve o fluxograma de um código interativo baseado no personagem Sanji, do anime *One Piece*, que tem que tomar decisões em uma aventura culinária. O código permite ao jogador fazer escolhas em diversos momentos, e o fluxograma reflete as possíveis alternativas e resultados. O objetivo deste trabalho é apresentar o fluxo de controle das decisões do programa, que se desdobram em diferentes resultados e ações com base nas escolhas do jogador.

\textbf{Local:} Escola Politécnica de Pernambuco - UPE/POLI

\textbf{Órgão Financiador:} N/A

\textbf{Caracterização:} Modelagem, Projeto e Implementação de Software em Linguagem \texttt{C}
\end{abstract}


%%%%%%%%%%%%%%%%%%%%%%%%%%%%%%%%%%%%%%%%%%%%%%%%%%%%%%%%%%%%%%%%%%%%%%%%%%%%%%%%%%%%%%%%
% artigo %%%%%%%%%%%%%%%%%%%%%%%%%%%%%%%%%%%%%%%%%%%%%%%%%%%%%%%%%%%%%%%%%%%%%%%%%%%%%%%

\section{Introdução}

Neste projeto, iremos apresentar o fluxograma de um código interativo baseado em um cenário envolvendo o personagem Sanji do anime *One Piece*. O código descreve uma aventura culinária na qual o jogador deve tomar decisões que impactam o desenvolvimento da história.

O fluxograma é a representação visual dessas escolhas e seus resultados, fornecendo uma visão clara do processo lógico do programa. O código será implementado na linguagem \texttt{C}, e o fluxograma ajudará na compreensão de suas estruturas condicionais e de controle.

%%%%%%%%%%%%%%%%%%%%%%%%%%%%%%%%%%%%%%%%%%%%%%%%%%%%%%%%%%%%%%%%%%%%%%%%%%%%%%%%%%%%%%%%
% seção de fluxograma %%%%%%%%%%%%%%%%%%%%%%%%%%%%%%%%%%%%%%%%%%%%%%%%%%%%%%%%%%%%%%%%%%%

\section{Fluxograma}

% adicionar o fluxograma aqui
\begin{tikzpicture}
    % Definindo os nós (blocos)
    \node (inicio) [startend] {Inicio};
    \node (txta) [print, below of=inicio] {Aventura começa: Sanji encontra ingredientes};
    \node (inp1) [input, below of=txta] {Escolha 1: Cogumelos ou Frutas?};
    \node (tot1) [process, below of=inp1] {Escolha 1: Cogumelos - Verificar o som};
    \node (maior) [decision, below of=tot1] {Som: investigar ou continuar?};
    \node (sim) [print, right of=maior, node distance=4cm] {Encontrou receita secreta};
    \node (nao) [print, below of=maior, node distance=2.4cm] {Armado pela Marinha, capturado};
    \node (fim1) [startend, below of=nao] {Fim da Aventura};

    % Fluxo para a segunda escolha de objetos
    \node (txtb) [print, below of=fim1] {Escolha 2: Frigideira ou Chapéu?};
    \node (inp2) [input, below of=txtb] {Escolha 2: Frigideira ou Chapéu};
    \node (tot2) [process, below of=inp2] {Escolha 2: Frigideira - Testar ou levar?};
    \node (maior2) [decision, below of=tot2] {Testar ou levar?};
    \node (sim2) [print, right of=maior2, node distance=4cm] {A frigideira quebra};
    \node (nao2) [print, below of=maior2, node distance=2.4cm] {Preservação correta};
    \node (fim2) [startend, below of=nao2] {Fim da Aventura};

    % Desenhando as setas
    \path [line] (inicio) -- (txta);
    \path [line] (txta) -- (inp1);
    \path [line] (inp1) -- (tot1);
    \path [line] (tot1) -- (maior);
    \path [line] (maior) -- (sim);
    \path [line] (sim) -- (fim1);
    \path [line] (maior) -- (nao);
    \path [line] (nao) -- (fim1);
    \path [line] (fim1) -- (txtb);
    \path [line] (txtb) -- (inp2);
    \path [line] (inp2) -- (tot2);
    \path [line] (tot2) -- (maior2);
    \path [line] (maior2) -- (sim2);
    \path [line] (sim2) -- (fim2);
    \path [line] (maior2) -- (nao2);
    \path [line] (nao2) -- (fim2);
\end{tikzpicture}

%%%%%%%%%%%%%%%%%%%%%%%%%%%%%%%%%%%%%%%%%%%%%%%%%%%%%%%%%%%%%%%%%%%%%%%%%%%%%%%%%%%%%%%%
% seção de justificativa %%%%%%%%%%%%%%%%%%%%%%%%%%%%%%%%%%%%%%%%%%%%%%%%%%%%%%%%%%%%%%%

\section{Detalhamento dos Autores}

\subsection*{Discentes}

\begin{enumerate}
    \item \textbf{Arthur Rodrigues Rocha Lopes}
    \begin{description}
        \item [Email:] \email{arrl@poli.br}
        \item [GitHub:] \url{https://github.com/arte011}
    \end{description}

    \item \textbf{Dante De Sá Leitão Melo De Oliveira}
    \begin{description}
        \item [Email:] \email{dslmo@poli.br}
        \item [GitHub:] \url{https://github.com/itsdanteoliveira}
    \end{description}

    \item \textbf{Lucas Pereira Silveira De Azevedo}
    \begin{description}
        \item [Email:] \email{lpsa@poli.br}
        \item [GitHub:] \url{https://github.com/LucasPSA3322}
    \end{description}
\end{enumerate}

\end{document}

